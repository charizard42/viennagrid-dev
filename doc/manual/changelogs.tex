
\chapter{Change Logs}

\subsection*{Version 2.0.0}
The ViennaGrid internals have been completely redesigned for higher flexibility.
Some rather significant adjustments to the user API were necessary.
\begin{itemize}
 \item Renamed the old \lstinline|domain_t| to \lstinline|mesh| in order to avoid ambiguities with the mathematical \emph{problem domain}.
 \item Replaced \lstinline|viennagrid::ncells<>()| with \lstinline|viennagrid::elements<>()| to obtain range objects.
 \item As a consequence of moving away from \lstinline|ncells<>|, now element tags are used instead of the topologic dimension to select elements.
 \item Added support for two dynamic element types: polygon and PLCs.
 \item Added support for neighbor iteration. This way one no longer needs to code the boundary/coboundary iterations by hand.
 \item Added support for multiple segmentations. This is a generalization of the old segment concept, where elements could be part of at most one segment.
 \item New algorithms: angles, intersection, scaling, seed point segmenting.
 \item Accessors are now consistently used for accessing quantities rather than ViennaData. This makes the implementation more generic and provides better support for user storage.
 \item New storage layer, re-wrote most of the internals.
\end{itemize}

\subsection*{Version 1.0.1}
This is a maintenance release, mostly fixing minor compilation problems on some compilers and operating systems. Other notable changes:
\begin{itemize}
  \item Added \lstinline|distance()| function for computing the distance between points, cell, etc.
  \item Voronoi quantities can now also be accessed in a more fine-grained manner: Volume and contributions for each cell attached to a vertex or edge.
  \item Added quantity transfer: Interpolates quantities on a $m$-cells can be transferred to $n$-cells. Both $m<n$ and $n>m$ are supported.
\end{itemize}

\subsection*{Version 1.0.0}
First release
