\chapter{Iterators} \label{chap:iterators}

The (possibly nested) iteration over elements of a mesh is one of the main ingredients for a plethora of algorithms.
Consequently, {\ViennaGrid} is designed such that these iterations can be carried out in a unified and flexible, yet efficient manner.

At the heart of the various possibilities is the concept of a \emph{range}. A range provides iterators for accessing a half-open interval \texttt{[first,one\_past\_last)} of elements and provides information about the number of elements in the range. However, a range does not 'own' the elements which can be accessed through it \cite{boost}.
Employing the range-concept, any iteration over elements in {\ViennaGrid} consists of two phases:
\begin{itemize}
 \item Initialize the range of elements over which to iterate.
 \item Iterate over the range using the iterators returned by the member functions \lstinline|begin()| and \lstinline|end()|.
\end{itemize}

For convenience, a range may also provide access to its elements using \lstinline|operator[]| (i.e.~random access) and thus allowing an index-based iteration. The conditions for random access availability will also be given in the following.

\TIP{A tutorial code can be found in \texttt{examples/tutorial/iterators.cpp}.}

\section{Elements in a Mesh or Segment}
As usual, the first step is to obtain the types for the range and the respective iterator.
To iterate over all elements of a \lstinline|mesh| of type \lstinline|MeshType|, the types can be obtained from the \lstinline|element_range| and \lstinline|iterator| meta-functions:
\begin{lstlisting}
 using namespace viennagrid;

 //non-const:
 typedef result_of::element_range<MeshType, ElementTag>::type ElementRange;
 typedef result_of::iterator<ElementRange>::type           ElementIterator;
\end{lstlisting}
For segments, all occurrences of \lstinline|MeshType| and \lstinline|mesh| have to be replaced by \lstinline|SegmentType| and \lstinline|segment| here and in the following.
If \lstinline|const|-access to the elements is sufficient, the meta-function \lstinline|const_element_range| should be used instead of \lstinline|element_range|.
For instance, the required types for a \lstinline|const|-iteration over vertices is given by
\begin{lstlisting}
 //const:
 typedef result_of::const_element_range<MeshType, vertex_tag>::type   ConstVertexRange;
 typedef result_of::iterator<ConstVertexRange>::type         ConstVertexIterator;
\end{lstlisting}

The next step is to set up a range object using the \lstinline|elements| function.
The general case of iteration over elements of a certain \lstinline|ElementTag| using some range type \lstinline|NCellRange| is handled by
\begin{lstlisting}
 NCellRange elements(mesh);
\end{lstlisting}
For the example of const-iteration over vertices, this results in
\begin{lstlisting}
 ConstVertexRange vertices(mesh);
\end{lstlisting}

Once the range is set up, iteration is carried out in the usual C++ STL manner:
\begin{lstlisting}
 for (ElementIterator it  = elements.begin();
                      it != elements.end();
                    ++it)
 { /* do something */}
\end{lstlisting}
For reference, the complete code for printing all vertices of a mesh is:
\begin{lstlisting}
 using namespace viennagrid;
 typedef result_of::const_element_range<MeshType, vertex_tag>::type
   ConstVertexRange;
 typedef result_of::iterator<ConstVertexRange>::type ConstVertexIterator;

 ConstVertexRange vertices(mesh);
 for (VertexIterator vit  = vertices.begin();
                     vit != vertices.end();
                   ++vit)
 {  std::cout << *vit << std::endl; }
\end{lstlisting}
It should be emphasized that this code snippet is valid for arbitrary geometric dimensions and arbitrary mesh configurations (and thus cell types). Inside a template function or template class, the \lstinline|typename| keyword needs to be added after each \lstinline|typedef|.



In some cases, e.g.~ for a parallelization using \OpenMP \cite{openmp}, it is preferred to iterate over all cells using an index-based for-loop rather than an iterator-based one.
If the range is either a vertex range of a mesh, or a cell range of a mesh or segment, this can be obtained by
\begin{lstlisting}
 ElementRange elements(mesh);
 for (std::size_t i=0; i<elements.size(); ++i)
 { do_something(elements[i]); }
\end{lstlisting}
It is also possible to use the range only implicitly:
\begin{lstlisting}
 for (std::size_t i=0;
                  i < viennagrid::elements<ElementTag>(mesh).size();
                ++i)
 {
   do_something(viennagrid::elements<ElementTag>(mesh)[i]);
 }
\end{lstlisting}
However, since the repeated construction of the range object can have non-negligible setup costs, the latter code is not recommended.

\NOTE{In {\ViennaGridversion}, \lstinline|operator[]| is not available for ranges obtained from a mesh other than vertex or cell ranges. For segments, \lstinline|operator[]| is only available for cell ranges.}

Shortcut meta-function are available for ranges, too.
For example, the vertex range and iterator type can be obtained directly via
\begin{lstlisting}
 typedef result_of::const_vertex_range<MeshType>::type ConstVertexRange;
 typedef result_of::iterator<ConstVertexRange>::type   ConstVertexIterator;
\end{lstlisting}
Similarly, instead of specifying the element tag directly, one can use the convenience functions
\lstinline|vertices()|, \lstinline|lines()|, \lstinline|edges()|, \lstinline|triangles()|, \lstinline|quadrilaterals()|, \lstinline|polygons()|, \lstinline|plcs()|, \lstinline|tetrahedra()|, and \lstinline|hexahedra()| for the quick extraction of ranges, e.~g.
\begin{lstlisting}
 for (std::size_t i=0; i<viennagrid::triangles(mesh).size(); ++i)
 {
   do_something(viennagrid::triangles(mesh)[i]);
 }
\end{lstlisting}

\section{Boundary Elements Iteration}
In addition to an iteration over all elements of a mesh or segment, it may be required to iterate over boundary elements such as all edges of a triangle.
% Instead of a general description using $n$-cells and $k$-cells, the more descriptive case of an iteration over all edges ($k=1$) of a \lstinline|cell| of type \lstinline|CellType| will be considered.

% \NOTE{In {\ViennaGridversion}, an iteration over all boundary elements of an element is not possible if the storage of boundary elements are disabled for elements, cf.~Section \ref{subsec:boundary-ncells-storage}. This restriction is expected to be relaxed in future versions of {\ViennaGrid}.}

As in the previous section, the range and iterator types are obtained from the \lstinline|element_range| and \lstinline|iterator| meta-functions:
\begin{lstlisting}
 //non-const:
 typedef result_of::element_range<CellType, line_tag>::type   EdgeOnCellRange;
 typedef result_of::iterator<EdgeOnCellRange>::type           EdgeOnCellIterator;
\end{lstlisting}
\lstinline|const|-ranges are obtained via \lstinline|const_element_range| instead of \lstinline|element_range|.
Mind that the first argument of \lstinline|element_range| denotes the enclosing entity (the cell) and the second argument denotes the boundary element tag (\lstinline|line_tag| or \lstinline|edge_tag| for an edge), and thus preserves the structure already used for the type retrieval for iterations on the mesh.

Iteration is then carried out in the same manner as for a mesh, with \lstinline|element| taking the role of the \lstinline|mesh| in the previous chapter.
The following snippet print all edges of an \lstinline|element|:
\begin{lstlisting}
 EdgeOnCellRange edges_on_cell(element);
 for (EdgeOnCellIterator eocit  = edges_on_cell.begin();
                         eocit != edges_on_cell.end();
                       ++eocit)
 {  std::cout << *eocit << std::endl; }
\end{lstlisting}

For all topological dimensions, an index-based iteration is possible provided that the storage of the respective boundary elements has not been disabled. The previous code snippet can thus also be written as
\begin{lstlisting}
 EdgeOnCellRange edges_on_cell(element);
 for (std::size_t i=0; i<edges_on_cell.size(); ++i)
 { do_something(edges_on_cell[i]); }
\end{lstlisting}
or
\begin{lstlisting}
 typedef viennagrid::edge_tag     EdgeTag;
 for (std::size_t i=0;
                  i < viennagrid::elements<EdgeTag>(element).size();
                ++i)
 { std::cout << viennagrid::elements<EdgeTag>(element)[i] << std::endl; }
\end{lstlisting}
The use of the latter is again discouraged for reasons of possible non-negligible repeated setup costs of the ranges involved.

% \NOTE{In {\ViennaGridversion}, an iteration over all $k$-cells of an $n$-cell is not possible if the storage of boundary $k$-cells is disabled for $n$-cells, cf.~Section \ref{subsec:boundary-ncells-storage}. This restriction is expected to be relaxed in future versions of {\ViennaGrid}.}

Finally, {\ViennaGrid} allows for iterations over the vertices of boundary elements of an element in the reference orientation imposed by the element, which is commonly required for ensuring continuity of a quantity along cell interfaces. Note that by default the iteration is carried out along the orientation imposed by the element in the way it is stored globally inside the mesh.
The correct orientation of vertices with respect to the hosting element is established by the free function \lstinline|local_vertex()|.
For instance, the vertices of a boundary element \lstinline|boundary_element| at the boundary of an element \lstinline|element| are printed in local orientation using the code lines
\begin{lstlisting}
 for (std::size_t i=0;
                  i < viennagrid::vertices(boundary_element).size();
                ++i)
   std::cout << viennagrid::local_vertex(element, boundary_element, i)
             << std::endl;
\end{lstlisting}
The use of \lstinline|local_vertex| can be read as follows: For the element \lstinline|element|, return the vertex of the boundary element \lstinline|boundary_element| at local position $i$.

\section{Coboundary Element Iteration}
A frequent demand of mesh-based algorithms is to iterate over so-called \emph{coboundary elements} of an element $T$.
The coboundary elements of an element $T$ are given by all elements of a mesh or segment, for which one of the boundary elements is $T$.
For example, the coboundary edges of a vertex $T$ are all edges where one of the two vertices is $T$.

In contrast to boundary elements, the number of coboundary elements of an element from the family of simplices or hypercubes is not known at compile time.
Another difference to the case of boundary elements is that the number of coboundary elements depends on the mesh or segment under consideration.
Considering the interface edge/facet connecting vertices $1$ and $4$ in the sample mesh from Fig.~\ref{fig:samplemesh}, the coboundary triangles within the mesh are given by the triangles $1$ and $2$.
However, within segment $0$, the set of coboundary triangles is given by the triangle $1$ only, while within segment $1$ the set of coboundary triangles consists of triangle $2$ only.
Thus, the use of segments can substantially simplify the formulation of algorithms that act on a subregion of the mesh only.

The necessary range types are obtained using the same pattern as in the two previous sections.
Assuming that a vertex type \lstinline|VertexType| is already defined, the range of coboundary edges as well as the iterator are obtained
using the \lstinline|coboundary_range| and \lstinline|iterator| meta-functions in the \lstinline|viennagrid| namespace:
\begin{lstlisting}
 //non-const:
 typedef result_of::coboundary_range< MeshType, VertexType, edge_tag
                                    >::type  EdgeOnVertexRange;
 typedef result_of::iterator<EdgeOnVertexRange>::type    EdgeOnVertexIterator;
\end{lstlisting}
The first argument to \lstinline|coboundary_range| is the context mesh or segment, the second is the reference element for the iteration, and the third argument is the element tag of the elements in the range.
A range of \lstinline|const|-edges is obtained using the \lstinline|const_coboundary_range| meta-function instead of the non-\lstinline|const| meta-function \lstinline|coboundary_range|.
Moreover, it shall be noted that an additional \lstinline|typename| keyword is required inside template functions and template classes.

To set up the range object, the \lstinline|coboundary_elements| function from the \lstinline|viennagrid| namespace is reused. Unlike in the previous sections, it requires two arguments
for setting up a coboundary range: The first argument refers to the enclosing container of elements and must be either a mesh or a segment and the second argument is the reference element.
The range holding all edges in the \lstinline|mesh| sharing a common vertex \lstinline|v| is thus set up as
\begin{lstlisting}
 EdgeOnVertexRange edges_on_v(mesh, v);
\end{lstlisting}

If the range should hold only the coboundary edges from a segment \lstinline|seg|, the above code line has to be modified to
\begin{lstlisting}
 EdgeOnVertexRange edges_on_v(seg, v);
\end{lstlisting}

An iteration over all edges is then possible in the usual STL-type manner. For example, all coboundary edges of $v$ in the range are printed using the code:
\begin{lstlisting}
 for (EdgeOnVertexIterator eovit = edges_on_v.begin();
                           eovit != edges_on_v.end();
                         ++eovit)
 { std::cout << *eovit << std::endl; }
\end{lstlisting}
One may also use an implicit form which does not set up the range explicitly:
\begin{lstlisting}
 for (EdgeOnVertexIterator eovit = viennagrid::coboundary_elements<
                                  VertexType, edge_tag>(mesh, v).begin();
                           eovit != viennagrid::coboundary_elements<
                                  VertexType, edge_tag>(mesh, v).end();
                         ++eovit)
 { std::cout << *eovit << std::endl; }
\end{lstlisting}

Random access, i.e.~\lstinline|operator[]| is available for all topological levels. Thus, the loop above may also be written as
\begin{lstlisting}
 for (std::size_t i=0; i<edges_on_v.size(); ++i)
 {
   std::cout << edges_on_v[i] << std::endl;
 }
\end{lstlisting}
or
\begin{lstlisting}
  for (std::size_t i=0;
                   i < viennagrid::coboundary_elements<
                          VertexType, edge_tag>(mesh, v).size();
                 ++i)
  {
    std::cout << viennagrid::coboundary_elements<
      VertexType, edge_tag>(mesh, v)[i] << std::endl;
  }
\end{lstlisting}
where the latter form is not recommended for reasons of overheads involved in setting up the temporary ranges.

Finally, it should be noted that coboundary information is not natively available in the mesh data structure.
If and only if for the first time the coboundary elements $C_n$ of an element $T$, are requested, an iteration over all elements of coboundary type of the mesh or segment with nested element $T$ boundary iteration is carried out to collect the topological information.
This results in extra memory requirements and additional computational costs, hence we suggest to use boundary iterations over coboundary iterations whenever possible.

\TIP{Prefer the use of boundary iterations over coboundary iterations to minimize memory footprint.}


\section{Neighbor Element Iteration}
In addition to boundary and coboundary iterations, {\ViennaGridversion} also supports iteration over neighboring elements of the same type.
Two elements are neighbors if they share a connector element of different type (with lower topologic dimension).
For example, one may iterate over all neighboring triangles of a reference triangle with vertex as connector element type.
As well as with coboundary iteration, a context mesh or segment has to be provided.


The necessary range types are obtained using the same pattern as with coboundary iteration:
Assuming that a triangle type \lstinline|TriangleType| is already defined, the range of neighbor triangles as well as the iterator are obtained
using the \lstinline|neighbor_range| and \lstinline|iterator| meta-functions in the \lstinline|viennagrid| namespace:
\begin{lstlisting}
 //non-const:
 typedef result_of::neighbor_range< MeshType, TriangleType, vertex_tag
                                   >::type  NeighborTriangleRange;
 typedef result_of::iterator<NeighborTriangleRange>::type   NeighborTriangleIterator;
\end{lstlisting}
The first argument to \lstinline|neighbor_range| is the context mesh or segment, the second is the reference element type for the iteration, and the third argument is the connector element tag.
A range of \lstinline|const|-edges is obtained using the \lstinline|const_neighbor_range| meta-function instead of the non-\lstinline|const| meta-function \lstinline|neighbor_range|.
Moreover, it shall be noted that an additional \lstinline|typename| keyword is required inside template functions and template classes.

To set up the range object, the \lstinline|neighbor_elements| function from the \lstinline|viennagrid| namespace is reused. Unlike in the previous sections, it requires two arguments
for setting up a neighbor range: The first argument refers to the enclosing container of elements and must be either a mesh or a segment and the second argument is the reference element.
The range holding all triangles in the \lstinline|mesh| sharing a common vertex with triangle \lstinline|t| is thus set up as
\begin{lstlisting}
 EdgeOnVertexRange neighbor_triangles_of_t(mesh, t);
\end{lstlisting}
If the range should hold only the neighbor triangles from a segment \lstinline|seg|, the above code line has to be modified to
\begin{lstlisting}
 EdgeOnVertexRange neighbor_triangles_of_t(seg, t);
\end{lstlisting}
An iteration over all triangles is then possible in the usual STL-type manner. For example, all neighbor triangles of \lstinline|t| in the range are printed using the code:
\begin{lstlisting}
 for (NeighborTriangleIterator ntit  = neighbor_triangles_of_t.begin();
                               ntit != neighbor_triangles_of_t.end();
                             ++ntit)
 { std::cout << *ntit << std::endl; }
\end{lstlisting}


One may also use an implicit form which does not set up the range explicitly:
\begin{lstlisting}
 for (NeighborTriangleIterator
      ntit  = viennagrid::neighbor_elements<TriangleType, vertex_tag
                                           >(mesh, t).begin();
      ntit != viennagrid::neighbor_elements<TriangleType, vertex_tag
                                           >(mesh, t).end();
    ++ntit)
 { std::cout << *ntit << std::endl; }
\end{lstlisting}

Random access, i.e.~\lstinline|operator[]| is available for all topological levels. Thus, the loop above may also be written as
\begin{lstlisting}
 for (std::size_t i=0; i<neighbor_triangles_of_t.size(); ++i)
 { std::cout << neighbor_triangles_of_t[i] << std::endl; }
\end{lstlisting}
or
\begin{lstlisting}
 for (std::size_t i=0;
                  i < viennagrid::neighbor_elements<
                        TriangleType, vertex_tag>(mesh, t).size();
                ++i)
 {
   std::cout << viennagrid::neighbor_elements<
     TriangleType, vertex_tag>(mesh, t)[i] << std::endl;
 }
\end{lstlisting}
where the latter form is not recommended for reasons of overheads involved in setting up the temporary ranges.
