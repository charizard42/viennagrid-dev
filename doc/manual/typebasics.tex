\chapter{Type Basics} \label{chap:typebasics}

A domain and all its topological elements as well as the underlying geometric space are specified in a common configuration class.
The setup of such a configuration class is explained in detail in Section \ref{sec:domain-configuration}.

\section{Domain Configuration} \label{sec:domain-configuration}

A domain is configured with a configuration class. This configuration class defines the topology as well as the geometry of a domain. {\ViennaGrid} supports a bunch of different configurations listed in the tables \ref{tab:configs-simplex}, \ref{tab:configs-hypercube} and \ref{tab:configs-special}.

\NOTE{Although {\ViennaGridversion} supports flexible configuration of the domain class, a convenient public interface is not provided so far. A next version will access this issue.}

\begin{table}[tb]
  \begin{center}
    \begin{tabular}{|l|l|}
      \hline
      \lstinline|line_1d|   &  Domain with lines and vertices in 1d  \\
      \hline
      \lstinline|line_2d|   &  Domain with lines and vertices in 2d  \\
      \hline
      \lstinline|line_3d|   &  Domain with lines and vertices in 3d  \\
      \hline 
      \lstinline|triangular_2d| &   Domain with triangles, lines and vertices in 2d  \\
      \hline
      \lstinline|triangular_3d| &   Domain with triangles, lines and vertices in 3d  \\
      \hline
      \lstinline|tetrahedral_3d| &   Domain with tetrahedrons, triangles, lines and vertices in 3d  \\
      \hline
    \end{tabular}
    \caption{Configurations of simplex domains}
    \label{tab:configs-simplex}
  \end{center}
\end{table}


\begin{table}[tb]
  \begin{center}
    \begin{tabular}{|l|l|}
      \hline
      \lstinline|quadrilateral_2d|   &  Domain with quadrilaterals, lines and vertices in 2d  \\
      \hline
      \lstinline|quadrilateral_3d|   &  Domain with quadrilaterals, lines and vertices in 3d  \\
      \hline
      \lstinline|hexahedral_3d|   &  Domain with hexahedrons, quadrilaterals, lines and vertices in 3d  \\
      \hline 
    \end{tabular}
    \caption{Configurations of hypercube domains}
    \label{tab:configs-hypercube}
  \end{center}
\end{table}


\begin{table}[tb]
  \begin{center}
    \begin{tabular}{|l|l|}
      \hline
      \lstinline|polygonal_2d|   &  Domain with polygons, lines and vertices in 2d  \\
      \hline
      \lstinline|polygonal_3d|   &  Domain with polygons, lines and vertices in 3d  \\
      \hline
      \lstinline|plc_2d|   &  Domain with PLCs, lines and vertices in 2d  \\
      \hline 
      \lstinline|plc_3d|   &  Domain with PLCs, lines and vertices in 3d  \\
      \hline 
    \end{tabular}
    \caption{Configurations of special domains}
    \label{tab:configs-special}
  \end{center}
\end{table}

\pagebreak

With a configuration, a domain can be defined using the meta-function \lstinline|domain|.

\begin{lstlisting}
 using namespace viennagrid;

 typedef config::trianglular_3d Config;
 
 // Type retrieval, method 1: use meta-function
 typedef result_of::domain<Config>::type     DomainType;
 
 // Type retrieval, method 1: use typedefed segmentation/segment
 typedef trianglular_3d_domain               DomainType;

 DomainType domain; //create the domain object
\end{lstlisting}

Using a domain type a segmentation and its segments can be defined.

\begin{lstlisting}
 using namespace viennagrid;
 
 // Type retrieval, method 1: use meta-function
 typedef result_of::segmentation<DomainType>::type        SegmentationType;
 typedef result_of::segmentation<SegmentationType>::type  SegmentType;
 
 // Type retrieval, method 1: use typedefed segmentaion/segment
 typedef trianglular_3d_segmentation      SegmentationType;
 typedef trianglular_3d_segment           SegmentType;

 DomainType domain; // create the domain object
 
 // segmentation needs the domain object as constructor argument
 SegmentationType segmentation(domain);
\end{lstlisting}

\pagebreak

\section{Elements and Handles} \label{sec:elements-and-handles}

A central topic with {\ViennaGrid} are elements. Elements types can be obtained by using the \lstinline|element| meta-function and the corresponding element tag

\begin{lstlisting}
 using namespace viennagrid;
 
 // obtain element type from domain type
 typedef result_of::element<DomainType, vertex_tag>::type         VtxType;
 // this meta function might fail if there is no tetrahedron in the domain
 typedef result_of::element<DomainType, tetrahedron_tag>::type    TetType;
 
 // obtain element type from another element
 typedef result_of::element<TetrahedronType, line_tag>::type      LineType;
 // the same element as above
 typedef result_of::element<TetrahedronType, edge_tag>::type      LineType;
 
 // this meta function will fail because there is no tetrahedron in a line
 typedef result_of::element<LineType, tetrahedron_tag>::type      TetType;
 
 // elements can also be obtained from segmentations or segments
 typedef result_of::element<SegmentationType, triangle_tag>::type TriType;
 typedef result_of::element<SegmentType, vertex_tag>::type        VtxType;
\end{lstlisting}

For each supported type there are also shortcut meta-functions available.

\begin{lstlisting}
 using namespace viennagrid;
 
 // obtain element type from domain type
 typedef result_of::vertex<DomainType>::type            VertexType;
 // this meta function might fail if there is no tetrahedron in the domain
 typedef result_of::tetrahedron<DomainType>::type       TetrahedronType;
 
 // obtain element type from another element
 typedef result_of::line<TetrahedronType>::type         LineType;
 
 // this meta function will fail because there is no tetrahedron in a line
 typedef result_of::tetrahedron<LineType>::type TetrahedronType;
 
 // elements can also be obtained from segmentations or segments
 typedef result_of::triangle<SegmentationType>::type    TriangleType;
 typedef result_of::vertex<SegmentType>::type           VertexType;
\end{lstlisting}

\pagebreak

Often elements are not accessed directly but through a handle instead. For example the boundary elements are stored using their handles within an element. Handles can also be simple obtained by the meta-function \lstinline|element| or by the shortcut meta-functions.

\begin{lstlisting}
 using namespace viennagrid;
 
 // obtain element handle type from domain type
 typedef result_of::handle<DomainType, vertex_tag>::type    VtxHandleType;
 typedef result_of::vertex_handle<DomainType>::type         VtxHandleType;
 // this meta function might fail if there is no tetrahedron in the domain
 typedef result_of::handle<DomainType, tetrahedron_tag>::type TetHandleType;
 typedef result_of::tetrahedron_handle<DomainType>::type    TetHandleType;
 
 // obtain element handle type from another element
 // (obtaining handles from another handle is not supported)
 typedef result_of::handle<TetrahedronType, line_tag>::type LineHandleType;
 typedef result_of::line_handle<TetrahedronType>::type      LineHandleType;
 // the same element as above
 typedef result_of::handle<TetrahedronType, edge_tag>::type LineHandleType;
 typedef result_of::edge_handle<TetrahedronType>::type      LineHandleType;
 
 // this meta function will fail because there is no tetrahedron in a line
 typedef result_of::handle<LineType, tetrahedron_tag>::type TetHandleType;
 typedef result_of::tetrahedron_handle<LineType>::type      TetHandleType;
 
 // elements can also be obtained from segmentations or segments
 typedef result_of::handle<SegmentationType, triangle_tag>::type TriHandleType;
 typedef result_of::triangle_handle<SegmentationType>::type TriHandleType;
 
 typedef result_of::handle<SegmentType>::type               VtxHandleType;
 typedef result_of::vertex_handle<SegmentType>::type        VtxHandleType;
\end{lstlisting}



