\chapter{Input/Output} \label{chap:io}

This chapter deals with the typical input and output operations: Reading a mesh from a file, and writing a mesh to a file.
In order not to give birth to another mesh file format, {\ViennaGrid} does not bring its own file format. 
Instead, the library mainly relies on the XML-based VTK \cite{VTK} file format \cite{VTKfileformat}.

\TIP{A tutorial code can be found in \texttt{examples/tutorial/io.cpp}.}

\TIP{Let us know about your favorite file format(s)!
 Send an email to our mailinglist: \texttt{viennagrid-support$@$lists.sourceforge.net}.
 It increases the chances of having a suitable reader and/or writer included in the next {\ViennaGrid} release.}


\section{Readers}
Due to the high number of vertices and cells in typical meshes,
a manual setup of a domain in code is soon inefficient. Therefore,
the geometric locations as well as topological connections are typically stored in mesh files.

Currently, {\ViennaGrid} supports only two file formats natively. However, readers for other file formats
can be easily added by the user when following the explanations for domain setup in Chapter \ref{chap:domainsetup}.
A different approach is to convert other file formats to one of the formats detailed in the following.

%\TIP{File reader contributions are always welcome!}

 \subsection{Netgen}
 The \texttt{.mesh}-files provided with \texttt{Netgen} \cite{netgen} can be imported directly.
 These files are obtained from \texttt{Netgen} from the \texttt{File->Export Mesh...} menu item. Note that only triangular and tetrahedral meshes are supported.

 To read a mesh from a \texttt{.mesh} file with name \lstinline|filename| to a \lstinline|domain|, the lines
 \begin{lstlisting}
 viennagrid::io::netgen_reader my_netgen_reader;
 my_netgen_reader(domain, segmentation, filename);
 \end{lstlisting}
 should be used. Note that the reader might throw an \lstinline|std::exception| if the file cannot be opened or if there is a parser error.

 The fileformat is simplistic: The first number refers to the number of vertices in the domain, then the coordinates of the vertices follow. After that, the number of cells in the domain is specified. Then, each cell is specified by the index of the segment and the indices of its vertices, each using index base $1$.
 For example, the \texttt{.mesh}-file for the sample domain in Fig.~\ref{fig:sampledomain} is:
 \begin{verbatim}
  6
  0 0
  1 0
  2 0
  2 1
  1 1
  0 1
  4
  1 1 2 6
  1 2 5 6
  2 2 3 5
  2 3 4 5
 \end{verbatim}

 \subsection{VTK}
 The VTK file format is extensively documented \cite{VTKfileformat} and allows to store mesh quantities as well.
 The simplest way of reading a VTK file \lstinline|my_mesh.vtu| is similar to the \texttt{Netgen} reader:
 \begin{lstlisting}
 viennagrid::io::vtk_reader<DomainType, SegmentationType>  my_vtk_reader;
 my_vtk_reader(domain, segmentation, "my_mesh.vtu");
 \end{lstlisting}
 Note that the domain type is required as template argument for the reader class, the segmentation type is optional. By default \lstinline|result_of::segmentation<DomainType>::type| is used.

 {\ViennaGrid} supports single-segmented \lstinline|.vtu| files consisting of one \lstinline|<piece>|.
 and always reads a single-segmented mesh to its first segment. If a segment does not already exist on the domain, one is created.

 For multi-segment meshes, the Paraview \cite{paraview} data file format \lstinline|.pvd| can be used, which is a XML wrapper holding information about the individual segments in \lstinline|.vtu| files only. Vertices that show up in multiple segments are automatically fused. Example meshes can be found in \texttt{examples/data/}.


 The VTK format allows to store scalar-valued and vector-valued data sets, which are identified by their names, on vertices and cells.
 These data sets are directly transferred to the {\ViennaGrid} domain using accessor as described in Chapter \ref{chap:data}.
 By default, data is stored using the data name string as key of type \lstinline|std::string|.
 Scalar-valued data is stored as \lstinline|double|, while vector-valued data is stored as \lstinline|std::vector<double>|.
%  For example, a scalar quantity \texttt{potential} on a vertex \lstinline|v| can be accessed and further manipulated using
%  \begin{lstlisting}
%   viennadata::access<std::string, double>("potential")(v)
%  \end{lstlisting}

%  For efficiency reasons, one may want to have data stored using either a different key, key type or data type.
%  This can be achieved using the following free functions in Tab.~\ref{tab:customizing-io} to set up the reader object accordingly.

There are two ways to obtain the data stored on vertices and cells: query the data after import or register an accessor/field before import.
Registering an accessor/field with an import is simply done by using the function \lstinline|add_*_data_on_*|, see Tab.~\ref{tab:customizing-io}

 \begin{lstlisting}
 std::vector<double> scalar_values;
 viennagrid::result_of::field< std::vector<double>, VertexType >::type scalar_values_field;
 
 std::vector< std::vector<double> > vector_values;
 viennagrid::result_of::field<
      std::vector< std::vector<double> >, CellType
    >::type vector_values_field;
 
 viennagrid::io::vtk_reader<DomainType>      my_vtk_reader;
 
 viennagrid::io::add_scalar_data_on_vertices(my_vtk_reader, scalar_values_field, "potential");
 viennagrid::io::add_vector_data_on_cells(my_vtk_reader, vector_values_field, "potential_vector_field");
 
 my_vtk_reader(domain, segmentation, "my_mesh.vtu");
 \end{lstlisting}
 
 

 \begin{table}[tb]
 \begin{center}
  \begin{tabular}{|l|l|}
   \hline
   Function name & Data description \\
   \hline
   \lstinline|add_scalar_data_on_vertices| & scalar-valued, vertex-based \\
   \lstinline|add_vector_data_on_vertices| & vector-valued, vertex-based \\
   \hline
   \lstinline|add_scalar_data_on_cells| & scalar-valued, cell-based \\
   \lstinline|add_vector_data_on_cells| & vector-valued, cell-based \\
   \hline
  \end{tabular}
 \end{center}
 \caption{Free functions in namespace \lstinline|viennagrid::io| for customizing reader and writer objects. The three parameters to each of these functions is the reader object, the accessor/field and the VTK data name.}
 \label{tab:customizing-io}
 \end{table}
 
 A list of all names identifying data read from the file can be obtained the functions
 \begin{center}
  \begin{tabular}{|l|l|}
   \hline
   Function name & Data description \\
   \hline
   \lstinline|get_scalar_data_on_vertices| & scalar-valued, vertex-based \\
   \lstinline|get_vector_data_on_vertices| & vector-valued, vertex-based \\
   \hline
   \lstinline|get_scalar_data_on_cells| & scalar-valued, cell-based \\
   \lstinline|get_vector_data_on_cells| & vector-valued, cell-based \\
   \hline
  \end{tabular}
 \end{center}

 The list of all scalar vertex data sets read is then printed together with the respective segment index as
 \begin{lstlisting}
using namespace viennagrid::io;
for (size_t i=0; i<get_scalar_data_on_vertices(reader).size(); ++i)
  std::cout << "Segment " << get_scalar_data_on_vertices(reader)[i].first 
            << ": "
            << get_scalar_data_on_vertices(reader)[i].second << std::endl;
 \end{lstlisting}
 
  
 
 \begin{table}[tb]
 \begin{center}
  \begin{tabular}{|l|l|}
   \hline
   Function name & Data description \\
   \hline
   \lstinline|vertex_scalar_field| & scalar-valued, vertex-based \\
   \lstinline|vertex_vector_field| & vector-valued, vertex-based \\
   \hline
   \lstinline|cell_scalar_field| & scalar-valued, cell-based \\
   \lstinline|cell_vector_field| & vector-valued, cell-based \\
   \hline
  \end{tabular}
 \end{center}
 \caption{Free functions in namespace \lstinline|viennagrid::io| for customizing reader and writer objects. The three parameters to each of these functions is the reader object, the accessor/field and the VTK data name.}
 \label{tab:io-field-obtaining}
 \end{table}
 
 \pagebreak
 
 After the import process a scalar/vector data field can be obtained with the member functions described in table Tab.~\ref{tab:io-field-obtaining}. In the next example a scalar field is obtained for the data values with VTK name \texttt{potential} and for the segment with ID $42$.
 
 \begin{lstlisting}
  viennagrid::result_of::field<std::vector<double>, VertexType> scalar_field =
    my_vtk_reader.cell_scalar_field( "potential", 42 );
 \end{lstlisting}
 
 
 
\section{Writers}
Since {\ViennaGrid} does not provide any visualization capabilities, the recommended procedure for visualization is 
to write to one of the file formats discussed below and use one of the free visualization suites for that purpose.

%\TIP{File writer contributions are always welcome!}

 \subsection{OpenDX}
 {\OpenDX} \cite{opendx} is an open source visualization software package based on IBM's Visualization Data Explorer.
 The writer supports either one vertex-based or one cell-based scalar quantity to be written to an {\OpenDX} file. OpenDX writer does not support segmentations.

 The simplest way to write a \lstinline|domain| of type \lstinline|DomainType| to a file \lstinline|"my_mesh.out"| is
 \begin{lstlisting}
  viennagrid::io::opendx_writer<DomainType> my_dx_writer;
  my_dx_writer(domain, "my_mesh.out");
 \end{lstlisting}
 To write quantities stored on the domain, the free functions from Tab.~\ref{tab:customizing-io} are used.
 For example, to visualize a scalar vertex-quantity of type \lstinline|double| stored in an accessor (cf.~Chapter \ref{chap:data}), the previous snippet is modified to
 \begin{lstlisting}
 using viennagrid::io;
 opendx_writer<DomainType> my_dx_writer;
 add_scalar_data_on_vertices(my_dx_writer,    // writer
                             value_accessor,  // some accessor
                             "some_name");    // ignored
 my_dx_writer(domain, "my_mesh.out");
 \end{lstlisting}
 Note that the data name provided as third argument is ignored for the {\OpenDX} writer.

 \NOTE{{\ViennaGrid} can write only one scalar quantity to an \texttt{OpenDX} file!}

 \subsection{VTK}
 A number of free visualization tools such as Paraview \cite{paraview} are available for the visualization of VTK files.
 The simplest way of writing to a VTK file is 
 \begin{lstlisting}
  viennagrid::io::vtk_writer<DomainType, SegmentationType> my_vtk_writer;
  my_vtk_writer(domain, segmentation, "outfile");
 \end{lstlisting}
 Each segment is written to a separate file, leading to \lstinline|"outfile_0.vtu"|, \lstinline|"outfile_1.vtu"|, etc. In addition,
 a Paraview data file \lstinline|"outfile_main.pvd"| is written, which links all the segments and should thus be used for visualization.
 
 If no segmentation is given, the file \lstinline|"outfile.vtu"| is written using the following code example
 \begin{lstlisting}
  viennagrid::io::vtk_writer<DomainType> my_vtk_writer;
  my_vtk_writer(domain, "outfile");
 \end{lstlisting}

 To write quantities stored on the domain to the VTK file(s), the free functions from Tab.~\ref{tab:customizing-io} are used.
 For example, to visualize a vector-valued cell-quantity of type \lstinline|std::vector<double>| stored in an accessor (cf.~Chapter \ref{chap:data}), the previous snippet is modified to
 \begin{lstlisting}
 using viennagrid::io;
 vtk_writer<DomainType> my_vtk_writer;
 add_vector_data_on_vertices(my_vtk_writer,        // writer
                             value_accessor,       // some accessor
                             "vtk_name");          // VTK name
  my_vtk_writer(domain, "outfile");
 \end{lstlisting}
 In this way, the quantity is written to all segments and values at the interface coincide.

 If discontinuities at the interfaces should be allowed, vertex or cell data may also be written per segment.
 
 As closing example, a quantity \lstinline|"jump"| with type \lstinline|double| is stored in an accessor.
 For a vertex \lstinline|v| at the interface of segments with indices $0$ and $1$, the values \lstinline|1.0| for the segment with index $1$ and \lstinline|2.0| for the segment with index $2$ are stored:
 \begin{lstlisting}
  std::vector<double> segment_0_values;
  std::vector<double> segment_1_values;
  
  viennagrid::result_of::field<std::vector<double>, VertexType>::type segment_0_values_accessor(segment_0_values);
  viennagrid::result_of::field<std::vector<double>, VertexType>::type segment_1_values_accessor(segment_1_values);
 
  segment_0_values_accessor(v) = 1.0;
  segment_1_values_accessor(v) = 2.0;
 \end{lstlisting}
 
 \pagebreak
 
 The quantity is added to the VTK writer as
 \begin{lstlisting}
  add_scalar_data_on_vertices(my_vtk_writer, segment_0, segment_0_values, "segment_data");
  add_scalar_data_on_vertices(my_vtk_writer, segment_1, segment_1_values, "segment_data");
 \end{lstlisting}

\TIP{A tutorial code using a VTK writer for discontinuous data at segment boundaries can be found in \texttt{examples/tutorial/multi-segment.cpp}.}
