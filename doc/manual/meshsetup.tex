\chapter{Mesh and Segment Setup} \label{chap:meshsetup}

This chapter explains how a {\ViennaGrid} mesh can be filled with cells. Since this
is a necessary step in order to do anything useful with {\ViennaGrid}, it is explained in detail in the following.
Existing file readers and writers are explained in Chapter \ref{chap:io}.

\TIP{A tutorial code can be found in \texttt{examples/tutorial/mesh\_setup.cpp}.}

In the following, the simple triangular mesh shown in Fig.~\ref{fig:samplemesh} will be set up.
Thus, the mesh type using the provided configuration class for two-dimensional triangular classes will be used:
\begin{lstlisting}
 using namespace viennagrid;

 typedef config::triangular_2d                      ConfigType;
 typedef result_of::mesh<ConfigType>::type          MeshType;
 typedef result_of::segmentation<MeshType>::type    SegmentationType;
 typedef result_of::segment<SegmentationType>::type SegmentType;

 // The mesh to be set up in the following
 MeshType mesh;

 // The segmentation for the example mesh in Figure 4.1
 SegmentationType segmentation(mesh);

 // Segment 0, the left one
 SegmentType segment_0 = segmentation.make_segment();
 // Segment 1, the right one
 SegmentType segment_1 = segmentation.make_segment();
\end{lstlisting}
Bear in mind that an additional \lstinline|typename| needs to be put after \lstinline|typedef| when using the meta-functions in namespace \lstinline|result_of| if used inside a template class or template function.
The mesh type is selected from the respective configuration class in namespace \texttt{config}.
The created mesh object will be filled with vertices and cells in the following.

\begin{figure}[tb]
\centering
 \includegraphics[width=0.65\textwidth]{figures/samplemesh.eps}
 \caption{An exemplary mesh for demonstrating the use of {\ViennaGrid}. Cell-indices are boxed.}
 \label{fig:samplemesh}
\end{figure}

\pagebreak

\section{Adding Vertices to a Mesh or Segment}
Since vertices carry geometric information by means of an associated point type, we first obtain the respective point type from the meta-function \lstinline|point<>|:
\begin{lstlisting}
 typedef viennagrid::result_of::point<MeshType>::type    PointType;
\end{lstlisting}
This already allows us to add the vertices shown in Fig.~\ref{fig:samplemesh} one after another to the mesh by using \lstinline|make_vertex|.
This function returns a handle to the created vertex.
The type of the vertex handle is obtained from either the general \texttt{handle}-meta-function as
\begin{lstlisting}
 typedef viennagrid::result_of::handle< MeshType,
                                        viennagrid::vertex_tag
                                      >::type    VertexHandleType;
\end{lstlisting}
or directly by using the convenience meta function \lstinline|vertex_handle|:
\begin{lstlisting}
 typedef viennagrid::result_of::vertex_handle<MeshType>::type 
   VertexHandleType;
\end{lstlisting}
One example for adding the first vertex is to push the respective point to the mesh using the member function \lstinline|make_vertex|, which also assigns an ID to the vertex:
\begin{lstlisting}
 PointType p(0,0);
 VertexHandleType vh0 = viennagrid::make_vertex(mesh, p);  // add vertex #0
\end{lstlisting} 

To push the next vertex, one can either reuse the existing vertex:
\begin{lstlisting}
 p[0] = 1.0;
 p[1] = 0.0;
 // add vertex #1
 VertexHandleType vh1 = viennagrid::make_vertex(mesh, p);
\end{lstlisting}

or directly construct the respective points in-place:
\begin{lstlisting}
 // add vertex #2
 VertexHandleType vh2 = viennagrid::make_vertex(mesh, PointType(2,0));
 // add vertex #3
 VertexHandleType vh3 = viennagrid::make_vertex(mesh, PointType(2,1));
 // ...
\end{lstlisting}

It is also possible to explicitly specify the ID for a new vertex:
\begin{lstlisting}
 typedef viennagrid::result_of::vertex<MeshType>::type   VertexType;
 typedef viennagrid::result_of::id<VertexType>::Type     VertexIDType;

 // add some vertex with ID 42
 VertexHandleType vh = viennagrid::make_vertex(mesh,
                                               VertexIDType(42),
                                               PointType(0,0));
\end{lstlisting}

If the geometric location of a particular vertex needs to be adjusted at some later stage, the free function \lstinline|point| can be used.
To e.g.~access the vertex with vertex handle \lstinline|vh4|,
\begin{lstlisting}
 viennagrid::point(mesh, vh4)
\end{lstlisting}
returns a reference to the respective element, which can then be manipulated accordingly.

Vertices can also be created through a segment \lstinline|seg1| directly:
\begin{lstlisting}
 VertexHandleType vh = viennagrid::make_vertex(seg1, PointType(1,1));
\end{lstlisting}
In this case the vertex is created in the underlying mesh and a handle is stored in the segment.


\section{Adding Cells to a Mesh or Segment}
In order to add cells to a mesh or segment, one first needs to extract its type.
This is conveniently possible via the \lstinline|cell| meta-function:
\begin{lstlisting}
 typedef viennagrid::result_of::cell<MeshType>::type              CellType;
\end{lstlisting}
Note that this is only a shortcut for the more general way of obtaining any mesh element type via the respective tag.
For cells, the above is equivalent to
\begin{lstlisting}
 using namespace viennagrid;
 typedef result_of::cell_tag<MeshType>::type          CellTag;
 typedef result_of::element<MeshType, CellTag>::type  CellType;
\end{lstlisting}

% However, it should be emphasized that this type definition does yield the cell type for e.g.~tetrahedral meshs.
% Thus, the more generic way of defining the cell type is to use the topological dimension stored in the cell tag of the mesh configuration class:
% \begin{lstlisting}
%  typedef ConfigType::cell_tag                                  CellTag;
%  typedef viennagrid::result_of::ncell<ConfigType,
%                                        CellTag::dim>::type     CellType;
% \end{lstlisting}
% These two lines reliable return the cell type for all types of meshs.
Note that an additional \lstinline|typename| is required if these lines are put inside a template class or template function.

\NOTE{If more than one element type has highest topological dimension within the mesh, this meta function will fail due to ambiguity.}


An overview of the generic procedure for adding a cell to a mesh or segment is the following:
  \begin{itemize}
   \item Set up an array holding the handles to the vertices \emph{in the mesh}. Do not use handles to vertices defined outside the mesh.
   \item Use \lstinline|make_element| to create the element within the mesh or segment
  \end{itemize}
Thus, the array of handles to vertices is created for a triangle as follows:
\begin{lstlisting}
 VertexHandleType cell_vertices[3];
\end{lstlisting}
Instead of hard-coding the number of vertices for a triangle, one can instead use
\begin{lstlisting}
static const std::size_t dim =
  viennagrid::boundary_elements<CellTag,
                                viennagrid::vertex_tag>::num;
VertexHandleType cell_vertices[dim];
\end{lstlisting}
resulting in code which will also work for e.g. quadrilateral meshes.

\NOTE{\lstinline|boundary_elements<CellTag,viennagrid::vertex_tag>::num| will only work on elements with static vertex count, e.g. triangles, tetrahedrons, ... This method will not work with polygons or PLCs}

Next, the vertex addresses for the first triangle are stored:
\begin{lstlisting}
 cell_vertices[0] = vh0; // vertex #0
 cell_vertices[1] = vh1; // vertex #1
 cell_vertices[2] = vh5; // vertex #5
\end{lstlisting}

\NOTE{Make sure that the correct cell orientation is used, cf.~Appendix!}

If the vertex handles are not available at this point, but the vertex IDs are known (this frequently occurs in mesh file readers) you can search them with their ID.
Keep in mind that depending on the underlying data structure this search might have linear runtime complexity.
\begin{lstlisting}
 typedef viennagrid::result_of::vertex<MeshType>::type   VertexType;
 typedef viennagrid::result_of::id<VertexType>::Type     VertexIDType;

 // vertex #0
 cell_vertices[0] = viennagrid::find_by_id(mesh, VertexIDType(0));
  // vertex #1
 cell_vertices[1] = viennagrid::find_by_id(mesh, VertexIDType(1));
  // vertex #5
 cell_vertices[2] = viennagrid::find_by_id(mesh, VertexIDType(5));
\end{lstlisting}

If you can ensure that the vertex with ID 0 was created first, the vertex with ID 1 was created second and so on, you can use direct random access:
\begin{lstlisting}
 typedef viennagrid::result_of::vertex<MeshType>::type   VertexType;
 typedef viennagrid::result_of::id<VertexType>::Type     VertexIDType;

 cell_vertices[0] = viennagrid::vertices(mesh)[0]; // vertex #0
 cell_vertices[1] = viennagrid::vertices(mesh)[1]; // vertex #1
 cell_vertices[2] = viennagrid::vertices(mesh)[5]; // vertex #5
\end{lstlisting}

Now we are ready to create the element within the mesh or segment. The generic function \lstinline|make_element| requires the element-to-create type as well as begin and end iterator of a vertex container.

\begin{lstlisting}
 viennagrid::make_element<CellType>(segment_0,
                                    cell_vertices, cell_vertices+3);
\end{lstlisting}

In our case the following shortcut function can also be used, the handles are passed directly.
\begin{lstlisting}
 viennagrid::make_triangle(segment_0, vh0, vh1, vh5);
\end{lstlisting}

As for vertices, cells have to be pushed in ascending order in order to get the correct IDs assigned. Note that the cell is always stored inside the mesh - a segment keeps a handle to the cell as well as its boundary cells only.

In the same way the other triangles are pushed to the respective segment. For triangle \lstinline|#3| in Fig.~\ref{fig:samplemesh}, the code is
\begin{lstlisting}
 viennagrid::make_triangle(segment_1, vh2, vh3, vh4);
\end{lstlisting}

As an alternative, the creation of elements by providing an explicit cell ID is also supported:

\begin{lstlisting}
 typedef viennagrid::result_of::id<CellType>::Type       CellIDType;

 viennagrid::make_element_with_id<CellType>(segment_0,
                                            cell_vertices, cell_vertices+3,
                                            CellIDType(42));
\end{lstlisting}
